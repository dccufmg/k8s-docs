\documentclass[12pt]{article}

\usepackage[utf8]{inputenc}
\usepackage{comment}

\usepackage[T1]{fontenc}
\usepackage{biblatex}
\addbibresource{referencias.bib}

\def\BibTeX{{\rm B\kern-.05em{\sc i\kern-.025em b}\kern-.08em
    T\kern-.1667em\lower.7ex\hbox{E}\kern-.125emX}}

\usepackage[portuguese]{babel}

\begin{document}
As doenças respiratórias juntamente com as cardiovasculares, cânceres e diabetes, são as doenças não transmissíveis (DNT) que mais matam hoje.
O Fórum das Sociedades Respiratórias Internacionais (FIRS) aborda a relação que existe nas doenças respiratórias e o meio ambiente. A queima de combustíveis, a poluição de fontes industriais e a fumaça do tabaco contribuem para a maioria das condições respiratórias \cite{firs}.

A infecção respiratória aguda do trato inferior é causa de aproximadamente 4 milhões de mortes por ano e são a principal causa de morte em crianças abaixo de 5 anos. É uma causa particularmente importante de morte em países de baixa e média renda. Entre as 30 causas mais comum de morte, a infecção é a quarta \cite{firs}.

As  doenças  respiratórias  são influenciadas por queimadas e os efeitos de inversões térmicas que concentram a poluição, impactando diretamente a qualidade do ar, principalmente nas áreas urbanas \cite{mudanças}.

A variação de respostas humanas relacionadas às mudanças  climáticas  parece  estar  diretamente  associada às  questões  de  vulnerabilidade  individual  e  coletiva. Variáveis  como  idade,  perfil  de  saúde,  resiliência  fisiológica e condições sociais contribuem diretamente para  as  respostas  humanas  relacionadas  às  variáveis climáticas.  Alguns  estudos também  apontam  que  alguns  fatores  que  aumentam a  vulnerabilidade  dos  problemas  climáticos  são  uma combinação  de  crescimento  populacional,  pobreza  e degradação  ambiental \cite{mudanças}.

As  condições  atmosféricas  podem  influenciar  o  transporte  de  microrganismos,  assim  como  de  poluentes oriundos de fontes fixas e móveis e a produção de pólen. Os efeitos das mudanças climáticas podem ser potencializados, dependendo das características físicas e químicas dos poluentes e das características climáticas como temperatura, umidade e precipitação. Essas características definem o tempo de residência dos poluentes na atmosfera, podendo ser transportados a  longas  distâncias  em  condições  favoráveis  de  altas temperaturas  e  baixa  umidade.  As alterações de temperatura, umidade e o regime de chuvas podem aumentar os efeitos das doenças respiratórias, assim como alterar as condições de exposição aos poluentes atmosféricos  \cite{mudanças}.

Em áreas urbanas alguns efeitos da exposição a poluentes  atmosféricos  são  potencializados  quando  ocorrem alterações  climáticas,  principalmente  as  inversões  térmicas.  Isto  se  verifica  em  relação  a  asma,  alergias,  infecções bronco-pulmonares e infecções das vias aéreas superiores  (sinusite),  principalmente  nos  grupos  mais susceptíveis, que incluem as crianças menores de 5 anos e indivíduos maiores de 65 anos de idade \cite{mudanças}.

Segundo a OMS, 50\% das doenças respiratórias crônicas e 60\% das doenças respiratórias agudas estão associadas à exposição a poluentes atmosféricos. A maioria dos estudos relacionando os níveis de poluição do ar com efeitos à saúde foram desenvolvidos em áreas metropolitanas, incluindo  as  grandes  capitais  da  Região  Sudeste  no  Brasil,  e mostram associação da carga de morbimortalidade por doenças  respiratórias,  com  incremento  de  poluentes atmosféricos,  especialmente  de  material  particulado. O tamanho da partícula, superfície e a composição química do material particulado determinam o risco para a saúde humana \cite{mudanças}.

As emissões gasosas e de material particulado para a atmosfera derivam principalmente de veículos, indústrias e da queima de biomassa. No Brasil, as fontes estacionárias  e  grandes  frotas  de  veículos  concentram-se nas áreas metropolitanas localizadas principalmente na Região Sudeste, enquanto a queima de biomassa ocorre em maior extensão e intensidade na Amazônia Legal \cite{mudanças}.

Alguns estudos evidenciam que a associação entre altas temperaturas e elevadas concentrações de poluentes  atmosféricos  pode  gerar  um  incremento  das  hospitalizações,  atendimentos  de  emergência,  consumo de medicamentos e taxas de mortalidade \cite{mudanças}.

As  doenças,  incluindo  as  respiratórias  que  são  grandes  responsáveis  por  mortes  e/ou incapacidade, podem retirar as pessoas do mercado de trabalho, ou diminuir seus rendimentos em razão de perda de produtividade. O aumento de 1\% nas internações por doenças respiratórias, que tem uma sobreposição com o Coronavírus em termos de faixa etária atingida e equipamentos necessários para  o  tratamento  (como  UTI  e  respiradores),  têm  impactos  negativos  no  PIB  per  capita  municipal  e  na renda  formal,    de  0,09\%  e  0,097\%  respectivamente \cite{impacto}.
\printbibliography
\end{document}