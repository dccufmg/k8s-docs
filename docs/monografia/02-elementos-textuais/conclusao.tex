% -----------------------------------------------------------------------------
% Conclusão
% -----------------------------------------------------------------------------

\chapter{Conclusão}
\label{chap:conclusao}

O desenvolvimento do presente trabalho possibilitou a avaliação de diferentes tipos de virtualização para comparação na implementação da plataforma de orquestração de cargas de trabalho de analise de dados em saúde. Através de profunda revisão bibliográfica foram identificados outras plataforma de orquestração de cargas de trabalhos e foi possível fazer uma avaliação critica baseada nos requisitos disponibilizados nas documentações oficiais, bem como a avaliação crítica do proposito ao qual o presente trabalho se propunha. Ainda na literatura, foi possível encontrar dados e informações a respeito dos impactos socio-econômicos resultantes da restrição orçamentária a pesquisa de uma forma geral. Em específico no caso desse trabalho para análise de dados em saúde, especialmente no auxilio de extração de informações relevantes de grande base de dados. O que torna possível a tomada de decisão em saúde pautada em dados e também auxilia a população a ter melhor noção, baseada em dados, da situação de saúde no qual se encontra e assim poder auditar os órgãos públicos responsáveis pela condução do SUS e políticas de saúde associadas.

\section{Trabalhos Futuros}
\label{sec:trabalhosFuturos}

Na continuação deste trabalho, o provisionamento do ambiente de testes com as devidas restrições e sua consequente avaliação será realizada, coletando dados de tempo de execução, taxa de utilização de memória e CPU, para comparação das duas propostas de sistemas virtualizados. Permitindo assim avaliar qual estratégia mais adequada para simulação de implementação e orquestração de dados.
No encerramento desse trabalho possibilita-se que o conhecimento desenvolvido ao longo do TCC possibilite a avaliação de implementação em maquinas físicas, podendo levar ao recrutamento de maquinas heterogêneas e de diversas configurações de hardware afim de criar um ou mais \emph{clusters} com capacidade suficiente para escalonar diversos tipos de cargas de trabalho, a exemplo da desse trabalho afim de possibilitar o processamento de dados massivos.

% \section{Considerações Finais}
% \label{sec:consideracoesFinais}

% As derradeiras palavras para encerramento do seu trabalho acadêmico.

% -----------------------------------------------------------------------------
% Observação: A norma ABNT estabelece que em qualquer tipo de trabalho
% acadêmico monográfico deve haver um capítulo de conclusão
% -----------------------------------------------------------------------------
