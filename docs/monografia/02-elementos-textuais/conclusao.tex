% -----------------------------------------------------------------------------
% Conclusão
% -----------------------------------------------------------------------------

\chapter{Conclusão}
\label{chap:conclusao}

Diversas tecnologias foram incorporadas e avaliadas para a orquestração das tarefas de processamento e importação de dados a serem utilizadas e \emph{cluster} Kubernetes\textregistered\ , o que se apresentou viável como alternativa econômica, para análise de grandes volumes de dados utilizando computadores comuns e reaproveitados. A eficiência energética não foi um tema abordado no trabalho, por entender-se que as máquinas reaproveitadas já estariam ligadas em laboratórios de informática da universidade, e por vezes subutilizadas. Outro caso que esse reaproveitamento pode amenizar, diz respeito ao ciclo de vida dessas máquinas no inventário de patrimônio em instituições públicas. A integralização desses equipamentos ao \emph{cluster} possibilita sua utilização mais eficiente (aproveitamento de máquinas ociosas) e prolonga a utilidade desses equipamentos (utilidade até descarte). Mesmo quando esses equipamentos estão defasados, ao integrar o \emph{cluster} podem receber cargas de trabalho mais propícias às suas configurações. 

O provisionamento do \emph{cluster} apresentou um tempo de configuração adequado para o projeto, não representando um percentual muito extenso do tempo disponível para realização desse trabalho. Sendo a maior parte do tempo gasto para finalização das configurações, relativos ao estudo das ferramentas utilizadas e sua correta configuração e integração. 

Sobre a avaliação de desempenho, o tempo de execução da importação e processamento dos dados foi adequado ao volume de dados apresentado, comparando cargas de trabalho semelhantes na experiência do autor. Porém, foram identificados diversos pontos de melhoria e otimizações possíveis. Essa identificação deixa claro que a integração de profissionais de múltiplas especialidades é essencial para utilizar essa alternativa como factível em estudos maiores e mais exigentes. Porém, essa alternativa se mostra promissora, especialmente sob o ponto de utilização de recursos subutilizados, com o processo de recrutamento através de configuração dinâmica de gerenciadores de configurações. 

Sobre o consumo de azitromicina, supõe-se que o estímulo ao seu consumo como opção de tratamento para a COVID-19, dentro do ‘kit COVID‘, possa ter influenciado o aumento das prescrições e, consequentemente, o aumento do consumo observado em 2020. O que indica que ações de saúde podem apresentar viés politico e influenciar decisões em saúde individuais entre a população. 

É importante que novas estratégias de avaliação de dados em saúde sejam propostas e validadas, especialmente na produção e viabilização do uso de ferramentas que contextualizem o cenário de investimento em pesquisa e educação do nosso país, de modo a tornar viável produzir soluções e pesquisas cada vez mais rapidamente no campo da saúde, ajudando a comunidade técnica e não técnica a tomarem melhores decisões embasadas em informações produzidas pela comunidade científica. 

Restrições orçamentárias não só impactam na velocidade, mas também na viabilização de projetos científicos que se valem de tecnologias. Na análise e processamento de dados públicos, devido ao volume de dados produzidos no Brasil, isso se torna um empecilho e projetos como esse se tornam ainda mais necessários.

O desenvolvimento do presente trabalho possibilitou a avaliação da aplicação de orquestração de cargas de trabalho de análise de dados de prescrições de azitromicina, advindo de uma base consideravelmente grande. Por meio da revisão bibliográfica foram identificadas outras plataformas de orquestração de cargas de trabalhos e foi possível fazer uma avaliação critica baseada nos requisitos disponibilizados nas documentações oficiais, bem como a avaliação crítica do propósito ao qual o presente trabalho se propunha. Ainda na literatura, foi possível encontrar dados e informações a respeito dos impactos socioeconômicos resultantes da restrição orçamentária a pesquisa de uma forma geral.  O que torna possível a tomada de decisão em saúde pautada em dados e também auxilia a população a ter melhor noção, baseada em dados, da situação de saúde onde se encontra e assim poder auditar os órgãos públicos responsáveis pela condução do SUS e políticas de saúde associadas.

\section{Trabalhos Futuros}
\label{sec:trabalhosFuturos}

O estudo de estratégias de otimização do dimensionamento de recursos e avaliação comparativa de mais tecnologias podem tornar ainda mais plausível a utilização de computadores ‘desktops’ subutilizados nas universidades para provisionamento de ambientes de análise de grande massas de dados. Permitindo assim propor estratégias e ferramental indicados como solução para essas análises, bem como uma estratégia viável de utilização de um conjunto de computadores quando ociosos. Como trabalhos futuros aponta-se a avaliação de algorítimos que capturem as métricas das execuções e avaliem a melhor configuração de recursos para cada task, visando otimizar tempo de execução e um objetivo de taxa de utilização do cluster. Outro trabalho possível de ser realizado é a pré-configuração de clusters em todos os laboratórios de informática e bibliotecas das universidades públicas, sendo estes federados a um \emph{cluster} mestre, que mediante a restrições de horário poderia distribuir tarefas aos seus federados, utilizando todos os computadores das universidades em caso de necessidade. Outro trabalho proposto seria um painel para submissão de atividades de análises para processamento em um \emph{cluster} central das universidades por outros departamentos, programas de pesquisas e instituições utilizando um conceito de cloud privada, porém em kubernetes semelhante à ideia proposta pela plataforma OpenShift da RedHat.
% \section{Considerações Finais}
% \label{sec:consideracoesFinais}

% As derradeiras palavras para encerramento do seu trabalho acadêmico.

% -----------------------------------------------------------------------------
% Observação: A norma ABNT estabelece que em qualquer tipo de trabalho
% acadêmico monográfico deve haver um capítulo de conclusão
% -----------------------------------------------------------------------------
