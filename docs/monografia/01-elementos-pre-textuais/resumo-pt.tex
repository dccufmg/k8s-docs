% -----------------------------------------------------------------------------
% Resumo
% -----------------------------------------------------------------------------

\begin{resumo}

    A análise de grandes volumes de dados, por vezes, requer um ou mais computadores de alto desempenho para tornar viável a extração de informações. Na área da pesquisa, principalmente em instituições públicas, há uma restrição orçamentária que, nos últimos anos, apresentou uma diminuição crescente dos recursos disponibilizados, inviabilizando a aquisição de computadores que possibilitam tais análises. Uma alternativa à necessidade de computadores de alto desempenho é o uso de técnicas de computação distribuída. Assim, foi realizada a orquestração de recursos em \emph{cluster} Kubernetes\textregistered\ para o processamento e a análise de dados. Além disso, avaliou-se a viabilidade dessa solução sob aspectos de desempenho, complexidade e viabilidade econômica, tendo como carga de trabalho a análise dos dados de consumo de azitromicina, no Brasil, entre os anos de 2014 e 2020.  Oito computadores foram utilizados na estruturação do cluster, somando 42 CPUs e 88 GB RAM. Para facilidade de configuração, mantendo a segurança dos clusters, todos foram configurados com chaves SSH para acesso remoto. A utilização de desktops na composição do \emph{cluster} - que continuará disponível para uso no Departamento de Ciências da Computação - viabilizou o processamento de mais de 70 GB de dados de maneira distribuída, em 53 minutos, sem nenhum custo adicional com a aquisição de equipamentos para análise de dados. Quando aos dados de consumo da azitromicina, no Brasil, observou-se um aumento das taxas de prescrição por 1.000 habitantes ao longo da série temporal de 66,2 para 83,8 prescrições, bem como em quase todas as unidades federativas, com destaque para Minas Gerais (66,2 para 149,2 prescrições por 1.000 habitantes), Rondônia (37,5 para 109,4 prescrições por 1.000 habitantes) e Roraima (37,2 para 99,8 prescrições por 1.000 habitantes). Comportamento semelhante também foi observado, na análise comparativa entre os anos pré e durante a pandemia (2019 e 2020), com aumento de 59,9 para 83,8 prescrições por mil habitantes, no país. A realização desse trabalhou demonstrou que a utilização de \emph{cluster} Kubernetes\textregistered\ é uma alternativa promissora para análise de grandes volumes de dados, especialmente sob o ponto de vista de utilização de recursos subutilizados, com o processo de recrutamento através de configuração dinâmica de gerenciadores de configurações. E que, considerando os dados analisados, aparentemente a pandemia da COVID-19 - incluindo a divulgação da azitromicina nos kits COVID - pode ter influenciado no maior aumento do consumo desse medicamento.


    \textbf{Palavras-chave}: Kubernetes\textregistered. Virtualização. Contêineres. Hypervisor Tipo 2. Análise de dados.
\end{resumo}

% -----------------------------------------------------------------------------
% Escolha de 3 a 6 palavras ou termos que descrevam bem o seu trabalho. As palavras-chaves são utilizadas para indexação.
% A letra inicial de cada palavra deve estar em maiúsculas. As palavras-chave são separadas por ponto.
% -----------------------------------------------------------------------------
