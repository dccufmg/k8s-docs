% -----------------------------------------------------------------------------
% Resumo
% -----------------------------------------------------------------------------

\begin{resumo}
    Análise de grandes massas de dados, por vezes requer um, ou mais computadores de alto desempenho para tornar viável a extração de informações. Na área da pesquisa, principalmente de instituições publicas, há uma restrição orçamentaria que nos últimos anos apresentou uma diminuição do valor disponibilizado. Dessa forma, áreas de estudo, como a da saúde, que possuem um volume de dados considerável, vêm enfrentando problemas na aquisição desses equipamentos que habilitam essas análises. Uma alternativa a necessidade de computadores de auto desempenho é o uso de técnicas de computação distribuída. O objetivo deste trabalho visa realizar o estudo das ferramentas disponíveis no mercado e propor uma solução de implementação e comparação, para no trabalho posterior, comparar o desempenho de orquestração de recursos em \emph{cluster} de baixo custo em ambientes com virtualização completa e virtualização baseada em contêineres para o processamento e a análise de dados em saúde, como tendência de consumo de azitromicina entre os anos 2014 e 2021. Ainda nesse trabalho apresenta-se uma avaliação de impactos econômicos na processamento de dados e aspectos sociais de contribuição resultantes dessa pesquisa.
    % A comparação entre os dois tipos de virtualização, por meio da taxa de utilização de memória e processamento, além do tempo de execução indicarão qual estratégia mais viável para composição do \emph{cluster}. E poderão ser comparados com a excussão em um computador com a soma dos recursos computacionais do cluster.

    \textbf{Palavras-chave}: Kubernetes\textregistered. Virtualização. Contêineres. Hypervisor Tipo 2. Análise de dados.
\end{resumo}

% -----------------------------------------------------------------------------
% Escolha de 3 a 6 palavras ou termos que descrevam bem o seu trabalho. As palavras-chaves são utilizadas para indexação.
% A letra inicial de cada palavra deve estar em maiúsculas. As palavras-chave são separadas por ponto.
% -----------------------------------------------------------------------------
