% -----------------------------------------------------------------------------
% Resumo
% -----------------------------------------------------------------------------

\begin{resumo}
    A análise de grandes massas de dados, por vezes, requer um ou mais computadores de alto desempenho para tornar viável a extração de informações. Na área da pesquisa, principalmente de instituições públicas, há uma restrição orçamentária que, nos últimos anos, apresentou uma diminuição do valor disponibilizado. Dessa forma, áreas de estudo, como a da saúde, que possuem um volume de dados considerável, vêm enfrentando problemas na aquisição desses computadores que habilitam essas análises. Uma alternativa à necessidade de computadores de alto desempenho é o uso de técnicas de computação distribuída. O objetivo deste trabalho é realizar o estudo das ferramentas disponíveis no mercado e propor uma solução de implementação e comparação para avaliar a viabilidade sob aspectos de desempenho, complexidade e viabilidade econômica a orquestração de recursos em  \emph{cluster} de baixo custo em ambientes com virtualização baseada em contêineres para o processamento e a análise de dados em saúde, como tendência de consumo de azitromicina entre os anos 2014 e 2021. Avaliaram-se os impactos econômicos no processamento de dados e aspectos sociais de contribuição resultantes dessa pesquisa. A viabilidade de utilização de ‘desktops’ na composição d \emph{cluster} inviabilizou o processamento de mais de 70GB de dados de maneira distribuída o que comprova a viabilidade dessa estratégia para orquestração de processamento de dados e \emph{cluster} e baixo custo com equipamentos reaproveitados. Além disso, pode se entrega u \emph{cluster} kubernetes operacional e configurado com uma stack de tecnologia para processamento de dados  atual e utilizada no mercado.
    % A comparação entre os dois tipos de virtualização, por meio da taxa de utilização de memória e processamento, além do tempo de execução indicarão qual estratégia mais viável para composição do \emph{cluster}. E poderão ser comparados com a excussão em um computador com a soma dos recursos computacionais do cluster.

    \textbf{Palavras-chave}: Kubernetes\textregistered. Virtualização. Contêineres. Hypervisor Tipo 2. Análise de dados.
\end{resumo}

% -----------------------------------------------------------------------------
% Escolha de 3 a 6 palavras ou termos que descrevam bem o seu trabalho. As palavras-chaves são utilizadas para indexação.
% A letra inicial de cada palavra deve estar em maiúsculas. As palavras-chave são separadas por ponto.
% -----------------------------------------------------------------------------
