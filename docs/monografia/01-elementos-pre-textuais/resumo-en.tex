% -----------------------------------------------------------------------------
% Abstract
% -----------------------------------------------------------------------------

\begin{resumo}[Abstract]
    Analysis of large datasets sometimes requires one or more high-performance computers to make extracting information feasible. In the area of research, mainly in public institutions, there is a budget constraint that in recent years has shown a decrease in the amount available. Thus, areas of study, such as health, which have a considerable volume of data, have been facing problems in acquiring the equipment that enables these analyses. An alternative to the need for self-performing computers is the use of distributed computing techniques. The objective of this work is to carry out a study of the tools available on the market and propose an implementation and comparison solution, in order to compare the performance of low-cost resource orchestration in \emph{cluster} in environments with full virtualization and virtualization. based on containers for the processing and analysis of health data, such as azithromycin consumption trend between the years 2014 and 2021. This work also presents an assessment of economic impacts on data processing and social aspects of contribution resulting from this research.

    \textbf{Keywords}: Kubernetes. Virtualization. Containers. Hypervisor Type 2. Data analysis.
\end{resumo}

% -----------------------------------------------------------------------------
% O restante da formatação deve manter-se igual ao do resumo em português, i.e, um único parágrafo.
% -----------------------------------------------------------------------------
