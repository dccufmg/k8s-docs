% -----------------------------------------------------------------------------
% Capa
% -----------------------------------------------------------------------------

% -----------------------------------------------------------------------------
% ATENÇÃO:
% Caso algum campo não se aplique ao seu documento - por exemplo, em seu trabalho
% não houve coorientador - não comente o campo, apenas deixe vazio, assim: \campo{}
% -----------------------------------------------------------------------------

% -----------------------------------------------------------------------------
% Dados do trabalho acadêmico
% -----------------------------------------------------------------------------

\titulo{Análise de dados utilizando \emph{cluster} de baixo custo}
\subtitulo{tendências de consumo da
azitromicina no Brasil antes e durante a
pandemia da COVID-19}
%\title{Title in English}
\autor{Felipe Fonseca Rocha}
\local{Belo Horizonte}
\data{Julho de 2022} % Normalmente se usa apenas mês e ano

% -----------------------------------------------------------------------------
% Natureza do trabalho acadêmico
% Use apenas uma das opções: Tese (p/ Doutorado), Dissertação (p/ Mestrado) ou
% Projeto de Qualificação (p/ Mestrado ou Doutorado), Trabalho de Conclusão de
% Curso (Graduação)
% -----------------------------------------------------------------------------

\projeto{Trabalho de Conclusão de Curso II}

% -----------------------------------------------------------------------------
% Título acadêmico
% Use apenas uma das opções:
% - Se a natureza for Tese, coloque Doutor
% - Se a natureza for Dissertação, coloque Mestre
% - Se a natureza for Projeto de Qualificação, coloque Mestre ou Doutor conforme o caso
% - Se a natureza for Trabalho de Conclusão de Curso, coloque Bacharel
% -----------------------------------------------------------------------------

\tituloAcademico{Bacharel}

% -----------------------------------------------------------------------------
% Área de concentração e linha de pesquisa
% Observação: Indique o nome da área de concentração e da linha de pesquisa do Programa de Pós-graduação
% nas quais este trabalho se insere
% Se a natureza for Trabalho de Conclusão de Curso, deixe ambos os campos vazios
% -----------------------------------------------------------------------------

% \areaconcentracao{Modelagem Matemática e Computacional}
% \linhapesquisa{Sistemas Inteligentes}

% -----------------------------------------------------------------------------
% Dados da instituição
% Observação: A logomarca da instituição deve ser colocada na mesma pasta que foi colocada o documento
% principal com o nome de "logoInstituicao". O formato pode ser: pdf, jpf, eps
% Se a natureza for Trabalho de Conclusão de Curso, coloque em "programa' o nome do curso de graduação
% -----------------------------------------------------------------------------

\instituicao{Universidade Federal de Minas Gerais}
\programa{Curso de Engenharia de Sistemas}
%\programa{Curso de Engenharia de Computação}
\logoinstituicao{0.2}{./04-figuras/brasao-ufmg.jpg} % \logoinstituicao{<escala>}{<nome do arquivo>}

% -----------------------------------------------------------------------------
% Dados do(s) orientador(es)
% -----------------------------------------------------------------------------

\orientador{Ítalo Fernando Scotá Cunha}
%\orientador[Orientadora:]{Nome da orientadora}
\instOrientador{Universidade Federal de Minas Gerais}
